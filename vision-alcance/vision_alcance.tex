\documentclass{article}

\usepackage[spanish]{babel}

\usepackage[letterpaper,top=2cm,bottom=2cm,left=3cm,right=3cm,marginparwidth=1.75cm]{geometry}

\usepackage{amsmath}
\usepackage{graphicx}
\usepackage[colorlinks=true, allcolors=blue]{hyperref}

\title{Pruebas unitarias en Angular}
\author{Brian Sánchez Robles - S19004873}


\begin{document}
\date{}

\maketitle


\section{Visión}
Las pruebas unitarias son una técnica de prueba en la cual un desarrollador autoprueba módulos individuales del código realizado para determinar si existe algún problema con su código y es una técnica que sirve para una mayor calidad de código y una entrega más rápida de nuestros productos al evitar la mayor parte de fallos en el código escrito.

Algunas ventajas de las pruebas unitarias son: 
\begin{itemize}

\item Pruebe clases individuales de forma aislada
\item Puede simular todas las condiciones de error.
\item Los desarrolladores pueden ejecutarlos después de cada modificación de archivo
\item Muy rápido y sin descamación
\end{itemize}


\section{Alcance}
El programa escrito para el extraordinario de la Experiencia Educativa, Pruebas de Software, consiste en una calculadora con las cuatro operaciones básicas con números complejos. Este programa fue elaborado con Angular y Jasmine, con deploy en Firebase.

En este proyecto se espera que el código escrito cubra un total del 100\% del coverage sumary, obteniendo los siguientes resultados.

\begin{itemize}

\item Statements : 100\%
\item Branches   : 100\%
\item Functions  : 100\%
\item Lines      : 100\%
\end{itemize}
\end{document}